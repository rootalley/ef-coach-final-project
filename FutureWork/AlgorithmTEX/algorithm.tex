\documentclass[letterpaper]{jdf}

\usepackage{amsmath}
\title{Priority Matrix Algorithm}
\author{
  Steven Endres\\
  steven.endres@gatech.edu
  \and
  Marissa McHugh\\
  marissamchugh@gatech.edu
}
\date{November 10, 2024}

\begin{document}

\maketitle

\section{Algorithm Description}

This algorithm helps college students prioritize homework assignments based on their urgency and importance. Each assignment is positioned on a two-dimensional grid, where the \(x\)-axis represents the assignment's urgency score and the \(y\)-axis represents its importance score. Both scales are normalized to the range \([0, 1]\) to plot the data in a square region upon output.

\subsection{Input}

The algorithm takes the following inputs:

\begin{itemize}
    \item The set of courses \( C \) in which the student is currently enrolled, where each course \( c \) has:
    \begin{itemize}
        \item \( c_{id} \): A unique identifier for the course.
        \item \( c_{score} \): The student's current score in the course, expressed as a percentage in the range \([0, 100]\).
    \end{itemize}
    \item The set of assignments \( A \) from all of the student's courses, where each assignment \( a \) includes:
    \begin{itemize}
        \item \( a_{id} \): A unique identifier for the assignment.
        \item \( a_{course} \): The ID of the course to which the assignment belongs.
        \item \( a_{due} \): The due date and time of the assignment.
        \item \( a_{points} \): The maximum points possible for this assignment.
    \end{itemize}
    \item \( t_{now} \): The current date and time.
\end{itemize}

\subsection{Processing}

For each assignment \( a \in A \), the algorithm performs the following calculations:

\subsubsection{Urgency Calculation}

The Urgency, \( U(a) \), of each assignment represents the number of full days remaining until the assignment's due date. Items due in less than a day, as well as assignments already past due, will have \( U(a) = 0 \). \( U(a) \) is given by:
\[
U(a) = \max\left(0, \left\lfloor \frac{a_{due} - t_{now}}{1\ \text{day}} \right\rfloor \right)
\]

\subsubsection{Importance Calculation}

The Importance, \( I(a) \), of each assignment is derived as follows:

\begin{enumerate}
    \item \textbf{Initial Importance:} The initial importance of an assignment represents its weight in the final grade of the course. Let \( P(a_{course}) \) denote the total possible points in the course to which assignment \( a \) belongs. The initial importance \( I_{initial}(a) \) is given by:
    \[
    I_{initial}(a) = \frac{a_{points}}{P(a_{course})}
    \]

    \item \textbf{Course Importance:} If a student is performing poorly in a given course (relative to their other courses), we increase the importance of each assignment in that course, up to a maximum of five times its initial importance. Let \( S_{\max} \) be the student's highest current score across all courses in \( C \). Assuming the student's current score is nonzero, we define a scaling factor \( \alpha(a) \) for the assignment as:
    \[
      \alpha(x)=\min\left(5.0, \frac{S_{\max}}{c_{score}(a_{course})}\right)
    \]
    
    If the student's current score in a course is zero, we increase the importance of the assignment by our maximum of five times its initial importance.

    \item \textbf{Final Importance:} Calculate the final importance score \( I(a) \) as:
    \[
    I(a) = I_{initial}(a) \times \alpha(a)
    \]
\end{enumerate}

\subsubsection{Normalization}

To standardize the Urgency and Importance values, each score is normalized as follows:

\begin{itemize}
    \item Let \( U_{\max} \) be the maximum Urgency score across all assignments. The normalized Urgency \( U_{norm}(a) \) for assignment \( a \) is:
    \[
    U_{norm}(a) = \frac{U(a)}{U_{\max}}
    \]

    \item Let \( I_{\max} \) be the maximum Importance score across all assignments. The normalized Importance \( I_{norm}(a) \) for assignment \( a \) is:
    \[
    I_{norm}(a) = \frac{I(a)}{I_{\max}}
    \]
\end{itemize}

\subsection{Output}

The output can be used to plot an Eisenhower matrix, assisting students in visually prioritizing their assignments based on urgency and importance. For each \( a \in A \), it provides:

\begin{itemize}
    \item \( a_{id} \): The unique identifier for the assignment.
    \item \( U_{norm}(a) \): The normalized urgency score in the range \([0, 1]\).
    \item \( I_{norm}(a) \): The normalized importance score in the range \([0, 1]\).
\end{itemize}

\begin{center}
$\blacksquare$
\end{center}

\end{document}